\documentclass[9pt]{article}
\usepackage{graphicx}
\usepackage{float}
\usepackage{amsmath}
\usepackage{amsfonts}
\usepackage[brazilian]{babel}
\usepackage[utf8]{inputenc}
\usepackage[backend=biber]{biblatex}
\usepackage{csquotes}
%\usepackage{docmute}
\usepackage{array}
\usepackage{multicol}
\usepackage[bindingoffset=0.2in,%
            left=0.3in,right=0.3in,top=0.1in,bottom=0.1in,%
            voffset=-0.5in,footskip=1.5in]{geometry}
\usepackage{algorithm}
\usepackage[noend]{algpseudocode}
\usepackage[T1]{fontenc}

%\addbibresource{report.bib}

\newcommand{\fromeng}[1]{\footnote{do inglês: \textit{#1}}}
\newcommand{\tit}[1]{\textit{#1}}
\newcommand{\tbf}[1]{\textbf{#1}}
\newcommand{\ttt}[1]{\texttt{#1}}

\newcolumntype{C}[1]{>{\centering\let\newline\\\arraybackslash\hspace{0pt}}m{#1}}

\begin{document}

%\newgeometry{margin=1in}

\author{Erik Perillo, RA135582}
\date{\today}
\title{%
	{\small Projeto e análise de algoritmos II}\\
	{\huge Laboratório 1 - Blueprints}}
%\subtitle{Lel}
\maketitle

\makeatletter
\def\BState{\State\hskip-\ALG@thistlm}
\makeatother

%\begin{multicols}{2}

\section{Abordagem}
Dados grafo $B$ (do arquivo morto) e $A$ (nome borrado),
Duas condições devem ser satisfeitas para serem da mesma cidade:
\begin{enumerate}
	\item Todos os vértices de B devem estar em A.
		Basta fazer uma busca exaustiva como descrita no pseudo-código abaixo:
		\begin{algorithm}
		\caption{}\label{euclid}
		\begin{algorithmic}[1]
		\Procedure{Contem}{A, B}
		\State $\{V_A, E_A\} \gets A, \{V_B, E_B\} \gets B$
		\If {$|V_B| > |V_A|$} \Return false \EndIf
		\For{v in $V_B$}
			\If {v $\not\in$ $V_A$} \Return false \EndIf
		\EndFor
		\Return true
		\EndProcedure
		\end{algorithmic}
		\end{algorithm}	

	\item Para cada vértice $(u, v)$ em $B$, existe um passeio em 
		$A$ $P = \{u, w_1,\ldots,w_{n}, v\}$ tal que $\{w_1,\ldots,w_n\}$
		não estão em $B$.
		A ideia é checar, para toda aresta $(u, v)$ em $B$, se há um passeio que 
		satisfaz tais condições. \ttt{DFS-VISIT}~\cite{clrs} 	
		em $A$ a partir de $u$ pode 
		ser usado para produzir uma lista $\pi$ de pais. 
		Checa-se então essa lista, a partir de $v$, indo até $u$ (se existe
		passeio), verificando se os vértices do meio não estão em $B$.
			
		\begin{algorithm}
		\caption{}\label{euclid}
		\begin{algorithmic}[1]
		\Procedure{Novos-vertices-em-velhas-conexoes}{A, B}
		\State $\{V_A, E_A\} \gets A, \{V_B, E_B\} \gets B$
		\For{$(u, v)$ in $E_B$}
			\State $\pi =$ DFS-VISIT(A, u)
			\State $w = \pi\left[v\right]$
			\If {$w = NULL$} \Return false \EndIf
			\While{$w \neq u$}
				\If{$u \in B$} \Return false \EndIf
				\State $w = \pi\left[w\right]$
			\EndWhile
		\EndFor
		\Return true
		\EndProcedure
		\end{algorithmic}
		\end{algorithm}	
\end{enumerate}

\begin{multicols}{2}
\section{Complexidade}
Vamos definir $n_1, m_1$ o número de vértices e arestas de $B$, respectivamente,
e $n_2, m_2$ os mesmos de $A$.

\subsection{Algoritmo 1}
O primeiro algoritmo na linha 4 executa $O(n_1)$ vezes a linha 5, que por sua
vez pode ser implementada em $O(n_2)$. Nas outras linhas, tem tempo constante.
Assim, sua complexidade assintótica é $O(n_2n_1)$ e, como $n_2 \ge n_1$, temos
o resultado final de $O(n_{2}^2)$.

\subsection{Algoritmo 2}
\ttt{DFS-VISIT} na linha 4 executa em $O(m_2)$. O loop da linha 7 é executado 
em $O(m_2)$ também e a linha 8 pode ser implementada em $O(n_1)$. Todos os
passos descritos acima acontecem dentro do \tit{loop} da linha 3 que executa 
em $O(m_1)$. As outras linhas são em $O(1)$. Assim, o tempo de execução é
$O(m_1\left(m_2 + m_2n_1\right)) = O(m_1m_2 + m_1m_2n_1)$ Sabe-se que há
limites entre o número de nós e arestas. Por exemplo:
$$m \le \frac{n^2 - n}{2}, n \le 2m$$
Sendo m e n o número de arestas e vértices de um grafo simples qualquer.
Assim, $m_1m_2 + m_1m_2n_1 \le m_1n_{2}^2 + m_1\frac{n_2}{2}n_1 \le 
m_1n_{2}^2 + m_1n_{2}^2$

\end{multicols}

\end{document}
