\documentclass[7pt]{article}
\usepackage{graphicx}
\usepackage{float}
\usepackage{amsmath}
\usepackage{amsfonts}
\usepackage[brazilian]{babel}
\usepackage[utf8]{inputenc}
\usepackage[backend=biber]{biblatex}
\usepackage{csquotes}
%\usepackage{docmute}
\usepackage{array}
\usepackage{multicol}
\usepackage[bindingoffset=0.2in,%
            left=0.3in,right=0.3in,top=0.1in,bottom=0.1in,%
            voffset=-0.5in,footskip=1.5in]{geometry}
\usepackage{algorithm}
\usepackage{titling}
\usepackage[noend]{algpseudocode}
\usepackage[T1]{fontenc}

\addbibresource{report.bib}

\newcommand{\fromeng}[1]{\footnote{do inglês: \textit{#1}}}
\newcommand{\tit}[1]{\textit{#1}}
\newcommand{\tbf}[1]{\textbf{#1}}
\newcommand{\ttt}[1]{\texttt{#1}}

\newcolumntype{C}[1]{>{\centering\let\newline\\\arraybackslash\hspace{0pt}}m{#1}}

\setlength{\intextsep}{1.5pt}
%\setlength{\textfloatsep}{-1in}

\begin{document}

%\newgeometry{margin=1in}

\author{Erik Perillo, RA135582}
\date{\today}
\title{%\vspace{-2cm}%
	{\small Projeto e análise de algoritmos II}\\
	{\Large Laboratório 1 - Blueprints}}
%\subtitle{Lel}
\maketitle\vspace{-1.3cm}
\posttitle{\par\end{center}}

\makeatletter
\def\BState{\State\hskip-\ALG@thistlm}
\makeatother

%\begin{multicols}{2}

\section{Abordagem}
Dados grafo $B$ (do arquivo morto) e $A$ (nome borrado),
Duas condições devem ser satisfeitas para serem da mesma cidade:
\begin{enumerate}
	\item Todos os vértices de B devem estar em A.
		Basta fazer uma busca exaustiva como descrita no pseudo-código abaixo:
		\begin{algorithm}
		\caption{}\label{euclid}
		\begin{algorithmic}[1]
		\Procedure{Contem}{A, B}
		\State $\{V_A, E_A\} \gets A, \{V_B, E_B\} \gets B$
		\If {$|V_B| > |V_A|$} \Return false \EndIf
		\For{v in $V_B$}
			\If {v $\not\in$ $V_A$} \Return false \EndIf
		\EndFor
		\Return true
		\EndProcedure
		\end{algorithmic}
		\end{algorithm}	

	\item Para cada vértice $(u, v)$ em $B$, existe um passeio em 
		$A$ $P = \{u, w_1,\ldots,w_{n}, v\}$ tal que $\{w_1,\ldots,w_n\}$
		não estão em $B$.
		A ideia então é fazer uma busca em profundidade~\cite{clrs}
		para toda aresta $(u, v) \in B$ de $u$ até o possível momento em que 
		o vértice $v$ seja alcançado, sempre checando se os vértices na
		`'jornada'' não estão em $B$. O pseudo-código a seguir descreve.
			
		\begin{algorithm}
		\caption{}\label{euclid}
		\begin{algorithmic}[1]
		\Procedure{Novos-vertices-em-velhas-conexoes}{A, B}
		\State $\{V_A, E_A\} \gets A, \{V_B, E_B\} \gets B$
		%\State $vertexInB_A \gets \emptyset$
		\For{$(u, v)$ in $E_B$}
			\State $visited \gets \emptyset \cup u$
			\For{$w$ in $Adj_A\left[u\right]$}
				\If {$\tbf{not}~visited\left[w\right] \tbf{and}$
					Novos-vertices-em-caminho($w, v, B, visited$)} \Return true 
				\EndIf
			\EndFor
		\EndFor
		\Return false
		\EndProcedure
		\end{algorithmic}
		\end{algorithm}	

		\begin{algorithm}
		\caption{}\label{euclid}
		\begin{algorithmic}[1]
		\Procedure{Novos-vertices-em-caminho}{u, v, B, visited}
		\State $visited\left[u\right] \gets true$
		\If{u = v} \Return true \EndIf
		\If{$u \in B$} \Return false \EndIf
		\For{$w$ in $Adj_A\left[u\right]$}
			\If {$\tbf{not}~visited\left[w\right] \tbf{and}$
				Novos-vertices-em-caminho($w, v, B, visited$)} \Return true 
			\EndIf
		\EndFor
		\Return false
		\EndProcedure
		\end{algorithmic}
		\end{algorithm}	

\end{enumerate}

\begin{multicols}{2}
\section{Complexidade}
Vamos definir $n_1, m_1$ o número de vértices e arestas de $B$, respectivamente,
e $n_2, m_2$ os mesmos de $A$.

\subsection{Algoritmo 1}
O primeiro algoritmo na linha 4 executa $O(n_1)$ vezes a linha 5, que por sua
vez pode ser implementada em $O(n_2)$. Nas outras linhas, tem tempo constante.
Assim, sua complexidade assintótica é $O(n_2n_1)$. Como todo vértice de $B$ está
em $A$, $n_2 \ge n_1$ e temos o resultado final de $O(n_{2}^2)$.

\subsection{Algoritmos 2 e 3}
O \tit{loop} da linha 3 do algoritmo 2 é executado no máximo $m_1$ vezes.
O algoritmo 3 é executado no máximo uma vez para cada vértice de $A$ 
(pois depois o vértice é marcado como visitado), passando por suas adjacências. 
Assim, o tempo total passado chamando o Algoritmo 3 é 
$\sum\nolimits_{v \in A}{d(v)} = O(m_2)$ para cada iteração da linha 3 do 
algoritmo 2.
A checagem da linha 4 do algoritmo 3 pode ser guardada, sendo executada
no máximo $n_2$ vezes, levando $n_1$ quando acontece, tendo como tempo
$O(n_2n_1)$.
O total então é $O(m_1m_2 + n_1n_2)$.
Agora, sabe-se que há limites entre o número de nós e arestas:
$m \le \frac{n^2 - n}{2}$, m e n o número de arestas e vértices de um grafo 
simples. Assim, $m_1m_2 + n_1n_2 \le m_1n_{2}^2 + 2n_{2}^2, (n \ge 0)$
E o tempo é então $O(m_1n_{2}^2)$.

\subsection{Total}
O tempo total é a soma dos tempos, que dá $O(n_{2}^2 + m_1n_{2}^2) = 
O(m_1n_{2}^2)$.

\printbibliography

\end{multicols}

\end{document}
