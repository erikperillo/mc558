\documentclass[7pt]{article}
\usepackage{graphicx}
\usepackage{float}
\usepackage{amsmath}
\usepackage{amsfonts}
\usepackage[brazilian]{babel}
\usepackage[utf8]{inputenc}
\usepackage[backend=biber]{biblatex}
\usepackage{csquotes}
%\usepackage{docmute}
\usepackage{array}
\usepackage{multicol}
\usepackage[bindingoffset=0.0in,%
            left=0.9in,right=0.9in,top=0.5in,bottom=0.5in,%
            voffset=0in,footskip=1.5in]{geometry}
\usepackage{algorithm}
\usepackage{titling}
\usepackage[noend]{algpseudocode}
\usepackage[T1]{fontenc}

\addbibresource{report.bib}

\newcommand{\fromeng}[1]{\footnote{do inglês: \textit{#1}}}
\newcommand{\tit}[1]{\textit{#1}}
\newcommand{\tbf}[1]{\textbf{#1}}
\newcommand{\ttt}[1]{\texttt{#1}}

\newcolumntype{C}[1]{>{\centering\let\newline\\\arraybackslash\hspace{0pt}}m{#1}}

%\setlength{\intextsep}{1.5pt}
%\setlength{\textfloatsep}{-1in}

\begin{document}

%\newgeometry{margin=1in}

\author{Erik Perillo, RA135582}
\date{\today}
\title{%\vspace{-2cm}%
	{\small MC558 - Projeto e análise de algoritmos II - Unicamp}\\
	{\Large Laboratório 2 - Caminhos na vida}}
%\subtitle{Lel}
\maketitle\vspace{-0.5cm}
\posttitle{\par\end{center}}

\makeatletter
\def\BState{\State\hskip-\ALG@thistlm}
\makeatother

%\begin{multicols}{2}

\section{Abordagem}
O desafio é determinar o número de caminhos seguindo a regra das cores em
tempo linear no tamanho do grafo. 
A ideia para isso foi executar um \ttt{topological-sort}~\cite{clrs} e, em
seguida, do sorvedouro para a fonte, contar os caminhos possíveis para cada
vértice até o vértice destino \ttt{t}. 
No final, teremos os caminhos possíveis para cada vértice, inclusive
o vértice inicial \ttt{s}.

O pseudo-algoritmo a seguir descreve de forma abstraída a ideia.
\begin{algorithm}
\caption{}\label{Caminhos-na-vida}
\begin{algorithmic}[1]
	\Procedure{Caminhos-na-vida}{G, s, t}
	\State $\{V, E\} \gets G$
	\State sort = topological-sort(G)
	\For{u in $V$}
		\State $caminhos_{G}[u] \gets 0$
		\State $caminhos_{Y}[u] \gets 0$
		\State $caminhos_{R}[u] \gets 0$
	\EndFor
	\State $caminhos_{G}[t] \gets 1$
	\For{u in $sort$ \tbf{in descending order}}
		\For{(u, v) in Adj[u]}
			\If{(u, v) é verde} 
				$caminhos_{G}[u] \gets caminhos_{G}[u] + 
				caminhos_{G}[v] + caminhos_{Y}[v] + 
				caminhos_{R}[v]$ 
			\EndIf
			\If{(u, v) é amarelo} 
				$caminhos_{Y}[u] \gets caminhos_{Y}[u] + 
				caminhos_{G}[v] + caminhos_{Y}[v]$
			\EndIf
			\If{(u, v) é vermelho} 
				$caminhos_{R}[u] \gets caminhos_{R}[u] + 
				caminhos_{G}[v]$ 
			\EndIf
		\EndFor
	\EndFor
	\For{u in $V$}
		\State $caminhos[u] \gets \sum\nolimits_{c \in \{G,Y,R\}}{caminhos_c[u]}$
	\EndFor
	\Return $caminhos[s]$
	\EndProcedure
\end{algorithmic}
\end{algorithm}	

\begin{multicols}{2}
\section{Complexidade}
Seja $V$ o número de vértices do grafo e $E$ o número de arestas do mesmo.

A chamada a \ttt{topological-sort} na linha 3 do algoritmo foi implementada
decentemente e então leva o tempo já conhecido de $O(V + E)$.
O \tit{loop} da linha 4 leva $O(V)$.
O \tit{loop} da linha 9 acontece $V$ vezes e o loop da linha 10 é executado
para cada lista de adjacência do grafo apenas uma vez por vértice, 
sendo executado então $E$ vezes. 
Assim, o total de tempo desses loops aninhados é $O(V + E)$.
O \tit{loop} da linha 14 acontece $V$ vezes.
O tempo total então é $O(V+E) + O(V) + O(V+E) + O(V)$ o que resulta em $O(V+E)$.

\printbibliography

\end{multicols}

\end{document}
