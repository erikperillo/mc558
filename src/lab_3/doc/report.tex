\documentclass[7pt]{article}
\usepackage{graphicx}
\usepackage{float}
\usepackage{amsmath}
\usepackage{amsfonts}
\usepackage[brazilian]{babel}
\usepackage[utf8]{inputenc}
\usepackage[backend=biber]{biblatex}
\usepackage{csquotes}
%\usepackage{docmute}
\usepackage{array}
\usepackage{multicol}
\usepackage[bindingoffset=0.0in,%
            left=0.9in,right=0.9in,top=0.5in,bottom=0.5in,%
            voffset=0in,footskip=1.5in]{geometry}
\usepackage{algorithm}
\usepackage{titling}
\usepackage[noend]{algpseudocode}
\usepackage[T1]{fontenc}

\addbibresource{report.bib}

\newcommand{\fromeng}[1]{\footnote{do inglês: \textit{#1}}}
\newcommand{\tit}[1]{\textit{#1}}
\newcommand{\tbf}[1]{\textbf{#1}}
\newcommand{\ttt}[1]{\texttt{#1}}

\newcolumntype{C}[1]{>{\centering\let\newline\\\arraybackslash\hspace{0pt}}m{#1}}

%\setlength{\intextsep}{1.5pt}
%\setlength{\textfloatsep}{-1in}

\begin{document}

%\newgeometry{margin=1in}

\author{Erik Perillo, RA135582}
\date{\today}
\title{%\vspace{-2cm}%
	{\small MC558 - Projeto e análise de algoritmos II - Unicamp}\\
	{\Large Laboratório 3 - Cabos}}
%\subtitle{Lel}
\maketitle\vspace{-0.5cm}
\posttitle{\par\end{center}}

\makeatletter
\def\BState{\State\hskip-\ALG@thistlm}
\makeatother

%\begin{multicols}{2}

\section{Abordagem}
O desafio é determinar um modo de se conectar todos os pontos de comunicação
com a mínima quantidade de cabos.
A ideia é fazer uma conversão para um problema de se achar a árvore geradora
mínima de um grafo cujas arestas representam um \tit{link} entre dois pontos
de conexão, com peso igual à distância entre eles.
Após obtida a árvore, basta contar as distâncias de cabo trançado usadas
e as distâncias de cabo de fibra.

O pseudo-algoritmo a seguir descreve de forma abstraída a ideia.
\begin{algorithm}
\caption{}\label{Cabos}
\begin{algorithmic}[1]
	\Procedure{Cabos}{points, fiber-thresh}
	\For{$p_1, p_2 \in points: p_1 \neq p_2$}
        \State $(u, v).weight = dist(p_1, p_2)$
        \State $G = G \cup (u, v)$
	\EndFor
    \State $s \gets \ttt{any-vertex(G)}$
    \State $\pi \gets$ \ttt{PRIM-ALG(G, s)}
    \State $\ttt{fiber-cost} \gets 0$
    \State $\ttt{normal-cost} \gets 0$
    \For{$u \in G: \pi[u] \neq NULL$}
        \State $cost \gets (\pi[u], u).weight$
        \If{$cost > \ttt{fiber-thresh}$}
            \State $\ttt{fiber-cost} \gets \ttt{fiber-cost} + cost$
        \Else
            \State $\ttt{normal-cost} \gets \ttt{normal-cost} + cost$
        \EndIf
	\EndFor
    \Return $(\ttt{normal-cost}, \ttt{fiber-cost})$
    \EndProcedure
\end{algorithmic}
\end{algorithm}

\begin{multicols}{2}
\section{Complexidade}
Seja $V$ o número de vértices do grafo e $E$ o número de arestas do mesmo.

As linhas 2 a 4 do algoritmo iteram sobre todas as combinações de pontos
a fim de formar um grafo fortemente conexo. Assim, tem $V^2$ iterações.
O algoritmo para se obter uma árvore geradora mínima é o algoritmo de
\tit{Prim}~\cite{clrs}, na linha 6.
Sua complexidade pode ser $O(E\log{V})$, mas a biblioteca \ttt{STL} de
\ttt{C++} não se dispõe do método \ttt{DECREASE-KEY} para \ttt{min-heaps},
o que fez com que se criasse uma nova \ttt{min-heap} a cada iteração
sobre os vértices ainda na \ttt{min-heap}. Um vetor que guardava se
certo vértice ainda estava na \ttt{min-heap} foi usado para ajudar 
nas construções.
Isso não é tão grave, visto que a criação de uma \ttt{min-heap} leva tempo
$O(V)$. Assim, a complexidade do algoritmo de \tit{Prim} implementada foi:
$O(V)$ chamadas a \ttt{build-min-heap}, mantendo as $O(V)$ chamadas a
\ttt{extract-min} e $O(E)$ \tit{updates} das \tit{keys}.
Assim, seu tempo total é $O(V^2 + V\log{V} + E)$.
O \tit{loop} das linhas 9-14 leva $O(V + E)$ (o termo $E$ se devendo ao
fato que há uma busca pelo peso). O tempo total então é:
$$O(V^2 + V^2 + V\log{V} + E + V + E) = O(V^2 + V\log{V})$$.

\printbibliography

\end{multicols}

\end{document}
