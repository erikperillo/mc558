\documentclass[7pt]{article}
\usepackage{graphicx}
\usepackage{float}
\usepackage{amsmath}
\usepackage{amsfonts}
\usepackage[brazilian]{babel}
\usepackage[utf8]{inputenc}
\usepackage[backend=biber]{biblatex}
\usepackage{csquotes}
%\usepackage{docmute}
\usepackage{array}
\usepackage{multicol}
\usepackage[bindingoffset=0.0in,%
            left=0.9in,right=0.9in,top=0.5in,bottom=0.5in,%
            voffset=0in,footskip=1.5in]{geometry}
\usepackage{algorithm}
\usepackage{titling}
\usepackage[noend]{algpseudocode}
\usepackage[T1]{fontenc}

\addbibresource{report.bib}

\newcommand{\fromeng}[1]{\footnote{do inglês: \textit{#1}}}
\newcommand{\tit}[1]{\textit{#1}}
\newcommand{\tbf}[1]{\textbf{#1}}
\newcommand{\ttt}[1]{\texttt{#1}}

\newcolumntype{C}[1]{>{\centering\let\newline\\\arraybackslash\hspace{0pt}}m{#1}}

%\setlength{\intextsep}{1.5pt}
%\setlength{\textfloatsep}{-1in}

\begin{document}

%\newgeometry{margin=1in}

\author{Erik Perillo, RA135582}
\date{\today}
\title{%\vspace{-2cm}%
	{\small MC558 - Projeto e análise de algoritmos II - Unicamp}\\
	{\Large Laboratório 4 - Corrida Maluca}}
%\subtitle{Lel}
\maketitle\vspace{-0.5cm}
\posttitle{\par\end{center}}

\makeatletter
\def\BState{\State\hskip-\ALG@thistlm}
\makeatother

%\begin{multicols}{2}

\section{Abordagem}
%O desafio é ajudar o Dr. Feynman a fazer um percurso de corredores em um
%supermercado com uma pontuação abaixo da limite e que tenha o corredor
%com maior pontuação possível.
O problema da Corrida Maluca pode ser visto como o de encontrar um caminho
de $s$ a $t$ em um grafo orientado que tenha o peso não maior que $w_{limit}$
e com uma aresta de maior peso possível.
A ideia é checar todas as arestas do grafo procurando pela maior delas que
satisfaça a restrição de peso de um caminho de $s$ a $t$.
O menor caminho possível incluindo uma aresta $(u, v)$ é o
menor caminho de $s$ a $u$ e o menor caminho de $v$ a $t$ ligados pela aresta
$(u, v)$. Se algum destes caminhos não existir, definimos que o seu peso é
$\infty$.
O pseudo algoritmo abaixo explica de forma abstraida a ideia.
\begin{algorithm}
\caption{}\label{Cabos}
\begin{algorithmic}[1]
    \Procedure{CRAZYRACE}{G, s, t, $w_{limit}$}
    \State $w_{max} \gets -1$
    \State $G^T \gets transpose(G)$
    \State $d_{s} \gets dijkstra(G, s)$
    \State $d_{t} \gets dijkstra(G^T, t)$
    \For{$(u, v) \in G$}
        \State $w \gets (u, v).weight$
        \If{$w > w_{max}$ \tbf{and} $d_{s}[u] + w + d_{t}[v] \le w_{limit}$}
            \State $w_{max} \gets w$
        \EndIf
	\EndFor
    \Return $w_{max}$
    \EndProcedure
\end{algorithmic}
\end{algorithm}

\begin{multicols}{2}
\section{Complexidade}
O grafo foi implementado como uma lista de adjacências.
Seja $V$ o número de vértices do grafo e $E$ o número de arestas do mesmo.

A linha 3 do algoritmo consome tempo $O(V + E)$ pois basta cópia dos vértices
de $G$ com uma passagem pelas arestas.
O algoritmo de \tit{Dijkstra}~\cite{clrs} nas linhas 4 e 5 consome tempo
$O((V + E)lg(V))$ pois o mesmo foi implementado com uma \tit{min-heap}.
O \tit{loop} da linha 6 acontece em $O(V + E)$.
Assim, temos como tempo total $O(V + E + (V + E)lg(V)) = O((V + E)lg(V))$.

\printbibliography

\end{multicols}

\end{document}
