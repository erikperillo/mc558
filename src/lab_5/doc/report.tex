\documentclass[7pt]{article}
\usepackage{graphicx}
\usepackage{float}
\usepackage{amsmath}
\usepackage{amsfonts}
\usepackage[brazilian]{babel}
\usepackage[utf8]{inputenc}
\usepackage[backend=biber]{biblatex}
\usepackage{csquotes}
%\usepackage{docmute}
\usepackage{array}
\usepackage{multicol}
\usepackage[bindingoffset=0.0in,%
            left=0.9in,right=0.9in,top=0.5in,bottom=0.5in,%
            voffset=0in,footskip=1.5in]{geometry}
\usepackage{algorithm}
\usepackage{titling}
\usepackage[noend]{algpseudocode}
\usepackage[T1]{fontenc}

\addbibresource{report.bib}

\newcommand{\fromeng}[1]{\footnote{do inglês: \textit{#1}}}
\newcommand{\tit}[1]{\textit{#1}}
\newcommand{\tbf}[1]{\textbf{#1}}
\newcommand{\ttt}[1]{\texttt{#1}}

\newcolumntype{C}[1]{>{\centering\let\newline\\\arraybackslash\hspace{0pt}}m{#1}}

%\setlength{\intextsep}{1.5pt}
%\setlength{\textfloatsep}{-1in}

\begin{document}

%\newgeometry{margin=1in}

\author{Erik Perillo, RA135582}
\date{\today}
\title{%\vspace{-2cm}%
	{\small MC558 - Projeto e análise de algoritmos II - Unicamp}\\
	{\Large Laboratório 5 - Galeria de arte}}
%\subtitle{Lel}
\maketitle\vspace{-0.5cm}
\posttitle{\par\end{center}}

\makeatletter
\def\BState{\State\hskip-\ALG@thistlm}
\makeatother

\section{Abordagem}
O problema a se resolver é o de usar o mínimo número possível de guardas
para cobrir regiões arbitrárias quadriculadas.

Pode-se reduzir o problema ao problema do emparelhamento máximo:
Nós da bipartição são pontos onde deve-se ter um vigilante.
Para cada dois pontos adjacentes, há um nó para cada lado da bipartição.
As direções de adjacências garantem que o grafo resultante nunca tenha
ciclos ímpares e, portanto, sempre há como montar uma bipartição e
temos uma transformação injetora.

Achar um emparelhamento máximo nesse esquema é equivalente a achar o máximo
número possível de quadrados que não há ninguém mas são vigiados por outro,
o que dá então o número mínimo possível de guardas a se usar: é o número
máximo (área da região) menos o número de quadrados poupados que conseguimos.

O pseudo algoritmo abaixo explica de forma abstraida a transformação: Primeiro
da matriz das regiões para uma bipartição e depois para uma rede de fluxo.
\begin{algorithm}
\caption{}\label{Galeria}
\begin{algorithmic}[1]
    \Procedure{TRANSFORM}{M}
    \State $(S, T) \gets bipartition(M)$
    \State $C \gets s \cup t$
    \For{$(x, y)~in~adjacencies(M)$}
        \State $(u, v) \gets map\_to\_edge(M, x, y)$
        \If{$v \in S$}
            \State $swap(u, v)$
        \EndIf
        \State $C \gets C \cup (s, u)$
        \State $C \gets C \cup (u, v)$
        \State $C \gets C \cup (v, t)$
	\EndFor
    \State $G \gets V(C)$\\
    \Return $(G, C, s, t)$
    \EndProcedure
\end{algorithmic}
\end{algorithm}

\begin{multicols}{2}

\section{Complexidade}
O grafo foi implementado como uma lista de adjacências.
As regiões a serem vigiadas foram implementadas como uma matriz binária, com
1 se e só se o ponto $(i, j)$ é uma região alfa.
Seja $V$ o número de vértices de um grafo e $E$ o número de arestas do mesmo e
$l$ o número de linhas e $c$ o número de colunas da matriz das regiões.

A linha 2 é simbólica pois na prática pode-se determinar qual o lado
da bipartição de um nó só com valores booleanos.
O \tit{loop} da linha 4 é basicamente uma iteração sobre cada ponto da matriz,
checando ao seu redor para adjacências. O mapeamento de um ponto para um nó
da linha 5 é feito em tempo constante com uma matriz de mapeamento.
As linhas 8 e 9 checam se as arestas já estão no grafo de capacidades
antes de adicioná-las. Isso é feito em tempo constante com um vetor.
Assim, a transformação leva tempo $O(lc)$.
A rede retornada é a entrada do algoritmo \ttt{EDMONDS-KARP} que foi
implementado como esperado com \ttt{BFS} e então leva tempo $O(VE^2)$
com $V, E$ das capacidades.
Para se obter a resposta do problema original a partir do fluxo máximo,
simplesmente calcula-se: $n = |C| - 2 - |f|$, onde $f, C$ são o fluxo e
capacidade, respectivamente. O grafo de capacidades é construído de tal
forma que ele tenha $lc + 2$ nós: pontos que não são adjacências simplesmente
não se conectam a nada. Isso leva tempo constante.
Assim, o tempo final do algoritmo é $O(lc + VE^2) = O(l^3c^3)$ pois
$V, E = O(lc)$.

\printbibliography

\end{multicols}

\end{document}
